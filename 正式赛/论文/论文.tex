\documentclass{mcmthesis}

\mcmsetup{CTeX = true,   % 使用 CTeX 套装时,设置为 true
        tcn = 2215432, problem = E,
        sheet = true, titleinsheet = true, keywordsinsheet = true,
        titlepage = false, abstract = false}
\geometry{left=1in,right=0.75in,top=1in,bottom=1in}
\numberwithin{figure}{section}
\numberwithin{table}{section}
\numberwithin{equation}{section}
\usepackage{newtxtext}
\usepackage{lipsum}
\usepackage{palatino}
\usepackage{hyperref}
\usepackage{booktabs}
\usepackage{subfigure}
\usepackage{graphicx}
\usepackage{pythonhighlight}
\usepackage{indentfirst}%首段自动缩进
\usepackage{colortbl}
\usepackage{apacite}
\usepackage{natbib}
\usepackage{tablefootnote}
\usepackage{multicol}
%算法四部曲↓
\usepackage{algorithm}
\usepackage{algpseudocode}
\usepackage{amsmath}
\usepackage{algorithmicx}
% \usepackage{threeparttable}

\setlength{\parindent}{2em}
\title{123}
\setlength{\headheight}{15pt}

\begin{document}
\renewcommand{\algorithmicrequire}{\textbf{Input:}}  % Use Input in the format of Algorithm
\renewcommand{\algorithmicensure}{\textbf{Output:}} % Use Output in the format of Algorithm
\begin{abstract}



\begin{keywords}
\end{keywords}
\end{abstract}
\maketitle

\tableofcontents
  \thispagestyle{empty}
  \newpage
  \setcounter{page}{1}
%%
%%Generate the Memorandum, if it's needed.
%\memoto{\LaTeX{}studio}
%\memofrom{Liam Huang}
%\memosubject{Happy \TeX{}ing!}
%\memodate{\today}
%%\logo{\LARGE I'm pretending to be a LOGO!}
%\begin{memo}[Memorandum]
%  \lipsum[1-3]
%\end{memo}

\section{Introduction}

\subsection{Problem Restatement}
In order to cope with the tremendous threat imposed by climate change, human-beings should 
spare no effort to reduce the amount of greenhouse gases in the atmosphere by not only cutting
down the emission of the greenhouse gases, but also sequestering carbon from the atmosphere into
plants, soil and water. Considering that the carbon dioxide can be sequestered in both forests 
and wooden products, it's reasonable that more carbon will be stored by forests with the 
appropriate combination of the regrowth of younger forests and the wooden products.
Thus, forest managers are ought to deliberate about the balance between the value of forests 
as living tress to grow and absorb the carbon and the value of forests harvested as wooden products.
What's more, the forest managers should not only consider the factors about forests such as 
type and age of forests, geography, topography, and benefits and lifespan of forest products,
but also the conservation and diversity of wild species, recreational uses and cultural considerations.
\par
The International Carbon Management (ICM) Collaboration has been established to guide the management
of forests all over the world under the consideration of different forest formation, climate, 
population, interests and values.
\begin{enumerate}
    \item [1.] Design a carbon sequestration model to calculate the amount of carbon dioxide 
    
    \item 
\end{enumerate}
\subsection{Overview of Our Work}

\newpage



\section{Assumptions and Justifications}
These are necessary assumptions for simplifying the model.
\begin{enumerate}
  \item [1.] 
\end{enumerate}


\section{Notations}

\renewcommand\arraystretch{1.5}

% \begin{table}[htpb!]
%   \centering
%   \caption{Notation Descriptions\citep{Vortex}} \label{Vortex_out}
%   \begin{tabular}{m{2.5cm}<{\centering}|m{12.5cm}<{\centering}}
%   \toprule[1.5pt]
%   \textbf{Symbol} & \textbf{Definition} \\ \hline
%   $ r $ & Innate rate of increase \\
%   Stoch r & The mean population growth rate experienced in the simulations, averaged 
%   across all years in which the population was extant. \\
%   %内廪增长率:给定的物理和生物的条件下,具有稳定的年龄组配的种群的最大瞬时增长率
%   N-extant  & Average extant population size\\ 
%   N-all  & Average population size \\
%   PE  & Probability of Extinction \\
%   GeneDiv  &  Genetic Diversity \\
%   TE & Time of Extinction(year) \\
%   medianTE & If at least 50\% of the iterations went extinct, 
%   the median time to extinction \\
%   SD & Standard Deviation \\
%   nAlleles & The mean number of alleles remaining within extant populations 
%   (from an original number equal to twice the number of founder individuals)\\
%   $ K $ & Carrying capacity \\ 
%   $ N_t $ & Size of Finless Porpoise population in the year of $ 1991 + t $ \\ 
%   \bottomrule[1.5pt]
%   \end{tabular}
%   \end{table}



\section{Introduction and Results of Models on Problem 1(a)}


\subsection{Result of Gray Forecast model}


\section{Sensitivity Test}

\section{Evaluation of Model}

\textbf{Strength:}


\section{Conclusions}



\newpage
\phantomsection\addcontentsline{toc}{section}{Policy Advice on Finless Porpoise Conservation}\tolerance=500
\memoto{Relavant Authorities}
\memofrom{MCM/ICM team XJ162}
\memodate{\today}

\begin{memo}[Policy Advice on Finless Porpoise Conservation]

  
\end{memo}




\newpage

%这一行是用来将Reference添加到目录的
\phantomsection\addcontentsline{toc}{section}{Refence}\tolerance=500

\bibliographystyle{apacite}
\bibliography{reference.bib}


\lhead{\small\sffamily \team}
\rhead{\small\sffamily Page \thepage\ }

\begin{appendices}

% \textbf{\textcolor[rgb]{0.98,0.00,0.00}{Input python source:}}
% \lstinputlisting[language=python]{./codes/verhulst.py}


% \textbf{\textcolor[rgb]{0.98,0.00,0.00}{Input matlab source:}}
% \lstinputlisting[language=Matlab]{./codes/lagrange_main.m}
% \lstinputlisting[language=Matlab]{./codes/lagrange.m}
% \lstinputlisting[language=Matlab]{./codes/ARIMA.m}
% \lstinputlisting[language=Matlab]{./codes/ca.m}
% \lstinputlisting[language=Matlab]{./codes/verhulst 2.m}

\end{appendices}


\end{document}

