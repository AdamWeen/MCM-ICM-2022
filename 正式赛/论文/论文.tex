\documentclass{mcmthesis}

\mcmsetup{CTeX = true,   % 使用 CTeX 套装时,设置为 true
        tcn = 2215432, problem = E,
        sheet = true, titleinsheet = true, keywordsinsheet = true,
        titlepage = false, abstract = false}
\geometry{left=0.75in,right=0.75in,top=1in,bottom=1in}
\numberwithin{figure}{section}
\numberwithin{table}{section}
\numberwithin{equation}{section}
\usepackage{newtxtext}
\usepackage{lipsum}
\usepackage{palatino}
\usepackage{hyperref}
\usepackage{booktabs}
\usepackage{subfigure}
\usepackage{graphicx}
\usepackage{pythonhighlight}
\usepackage{indentfirst}%首段自动缩进
\usepackage{colortbl}
\usepackage{apacite}
\usepackage{natbib}
\usepackage{tablefootnote}
\usepackage{multicol}
%算法四部曲↓
\usepackage{algorithm}
\usepackage{algpseudocode}
\usepackage{amsmath}
\usepackage{algorithmicx}
% \usepackage{threeparttable}

\setlength{\parindent}{2em}
\title{123}
\setlength{\headheight}{15pt}

\begin{document}
\renewcommand{\algorithmicrequire}{\textbf{Input:}}  % Use Input in the format of Algorithm
\renewcommand{\algorithmicensure}{\textbf{Output:}} % Use Output in the format of Algorithm
\begin{abstract}



\begin{keywords}
\end{keywords}
\end{abstract}
\maketitle

\tableofcontents
  \thispagestyle{empty}
  \newpage
  \setcounter{page}{1}
%%
%%Generate the Memorandum, if it's needed.
%\memoto{\LaTeX{}studio}
%\memofrom{Liam Huang}
%\memosubject{Happy \TeX{}ing!}
%\memodate{\today}
%%\logo{\LARGE I'm pretending to be a LOGO!}
%\begin{memo}[Memorandum]
%  \lipsum[1-3]
%\end{memo}

\section{Introduction}

\subsection{Problem Restatement}
In order to cope with the tremendous threat imposed by climate change, human-beings should 
spare no effort to reduce the amount of greenhouse gases in the atmosphere by not only cutting
down the emission of the greenhouse gases, but also sequestering carbon from the atmosphere into
plants, soil and water. Considering that the carbon dioxide can be sequestered in both forests 
and wooden products, it's reasonable that more carbon will be stored by forests with the 
appropriate combination of the regrowth of younger forests and the wooden products.
Thus, forest managers are ought to deliberate about the balance between the value of forests 
as living tress to grow and absorb the carbon and the value of forests harvested as wooden products.
What's more, the forest managers should not only consider the factors about forests such as 
type and age of forests, geography, topography, and benefits and lifespan of forest products,
but also the conservation and diversity of wild species, recreational uses and cultural considerations.
\par
The International Carbon Management (ICM) Collaboration has been established to guide the management
of forests all over the world under the consideration of different forest formation, climate, 
population, interests and values.
\begin{enumerate}
    \item [1.] Design a carbon sequestration model to calculate the amount of carbon dioxide 
    sequestered by a forest and its products, which also determines what kind of manage plan is
    most efficient at sequestering carbon.
    \item [2.] Develop a decision model consisting of various ways that forests are valued (including
    carbon sequestration) for forest managers to understand the best use of a forest. Consider 
    the following questions as well as those of your own:
    \begin{enumerate}
        \item [a.] What is the scope of the management plan that your decision model might suggest?
        \item [b.] Are there any conditions that the forests should not be harvested?
        \item [c.] Whether there is a transition point between management plans applicable to all forests?
        \item [d.] How can the characteristics of a particular forest and its location be used 
        to determine transition points between management plans?
    \end{enumerate}
    \item [3.] Apply your models to various forests. Identify a forest that your decision model 
    would suggest the inclusion of harvesting in its management plan.
    \begin{enumerate}
        \item [a.] How much carbon dioxide can be sequestered by this forest and its products in 100 years?
        \item [b.] What kind of forest management plan should be carried out for this forest? Why it's best?
        \item [c.] The best management plan is assumed to include a harvest interval of 10 
        years longer than current forest practices discussing strategies for transitioning 
        from existing to new schedules in a manner sensitive to the needs of forest managers 
        and all those who use forests.
    \end{enumerate}
    \item [4.] Some people think that we should never cut down any trees, but you've
    determined the forest which includes harvest in its management. Write a one-to-two-page
    non-technical newspaper articles to explain why your analysis including in the 
    management of the forest logging, rather than it remaining the same. Finally, your ariticle 
    should convince local community that it's the best decision for their forest.
\end{enumerate}

\newpage
\subsection{Overview of Our Work}

\section{Assumptions and Justifications}
These are necessary assumptions for simplifying the model.
\begin{enumerate}
  \item [1.] 
\end{enumerate}


\section{Notations}

\renewcommand\arraystretch{1.5}

\begin{table}[htpb!]
  \centering
  \caption{Notation Descriptions}
  \begin{tabular}{m{2.5cm}<{\centering}|m{12.5cm}<{\centering}}
  \toprule[1.5pt]
  \textbf{Symbol} & \textbf{Definition} \\ \hline
  $ \rm{DBH} $ & Diameter at Breast Height \\
  $ \beta_b $ & Conversion factor of coniferous trees\\
  $\beta_c$ & Conversion factor of broad leaf trees\\
  $ n $ & Total iteration years\\
  $ \lambda $ & Harvest rate\\
  $ M $ & Sequestered carbon mass in wooden products \\
  $ \rm{CarbonMass_f} $ & Sequestered carbon mass in forests \\
  $ \phi $ & Proportion in total product\\
  $ s $ & Scrap rate\\
  $ m $ & Decomposition rate \\

  \bottomrule[1.5pt]
  \end{tabular}
\end{table}



\section{Introduction and Results of Models on Problem 1(a)}


\begin{algorithm}[htbp]
    \caption{Binary Timber Volume Regression of Carbon Prediction Algorithm} % 名称
    \begin{algorithmic}[1]
      \Require
        Measurement of DBH, Height of tree ($ h $), conversion factor 
        $ \beta_c $  (coniferous) and $ \beta_b $ (broad leaf), Harvest rate $ \lambda(t) $ 
        \For{x in enumarate $ (\rm{DBH},h) $ }
        \For{k $ \gets 1$ to $ m $}
        $ P_k(x) = \frac{1}{2^kk!}\frac{d^k}{dx_k^k}(x^2-1)^k $
        \For{$ l \gets 1 $ to $m$ }

        \quad \quad Calculate $ \int_{-1}^{1}P_k(x)P_l(x)dx \gets $ Integral of Legendre Polynomials
        \If{$ k = l $}$ \int_{-1}^1P_k(x)P_l(x) = \frac{2x}{2i+1} $
        \Else $ \int_{-1}^1P_k(x)P_l(x) = 0 $   
        \EndIf 
        \EndFor
        \EndFor
        \EndFor
        
        \noindent Find $ (\rm{DBH}, h)^T $ to minimize  $ J = \int_a^b[f(\rm{DBH})+g(h)-2(x)]^2dx $
        \For{$ t \gets 1 $ to $ n $ }

        Regression of DBH and $ h $  

        $V_0 \gets$ Initiate

        $ V_t = (V_{t-1}+a\rm{DBH}^bh^c)\times $$Area\times Density(t) \times (1-\lambda(t)) $ 
        \EndFor

        \noindent$ Area = Area_b + Area_c $

        \noindent $ \rm{CarbonMass_f} $  $= \beta_cV_c+\beta_bV_b+9.6\times Area_c+3.4\times Area_b + 70\times Area$ 

        \Ensure
        Carbon Dioxide Quantity of Forest = $\frac{44}{12}\times\rm{CarbonMass_f}$ 
    \end{algorithmic}
  \end{algorithm}
  
  \begin{algorithm}[htbp]
    \caption{RBF Neural Network Fitting of wooden products for carbon sequestration Algorithm} % 名称
    \begin{algorithmic}[1]
        \Require
            $ \phi $, $ s $, $ m $, CarbonMass and $ V $ in Algorithm1, $ w $ as kinds wooden products, Standardized CarbonMass and $ V $  
        \For{$ t\gets1 $ to n}
            \For{$ i\gets1 $ to hidden\_dim}

                $ y_0 \gets Initiate $ 

                $ \hat{y_t} = \hat{y_{t-1}} + \phi_{it}V_i(t) $
            
                $ c_i\gets $ Sample CarbonMass
            
                $ \sigma_i\gets $ Z-score Normalization

                $ V_i(t) = e^{-\frac{||t-c_i||^2}{2\sigma_i^2}} $
            \EndFor

            $ M_0\gets Initiate $ 

            \For{$ j\gets 1$ to $ w $}

            $ M_t = M_{t-1}+s\lambda(t)V(t)\times \rm{CarbonMass} \times (\phi_j(t)m_j(t)) $ 
            \EndFor
        \EndFor
        \Ensure
        Carbon Dioxide Quantity of Wooden Products = $ \frac{44}{12} \times M_t$ 
    \end{algorithmic}
\end{algorithm}

\section{Sensitivity Test}

\section{Evaluation of Model}

\textbf{Strength:}


\section{Conclusions}



\newpage
\phantomsection\addcontentsline{toc}{section}{Policy Advice on Finless Porpoise Conservation}\tolerance=500
\memoto{Relavant Authorities}
\memofrom{MCM/ICM team XJ162}
\memodate{\today}

\begin{memo}[Policy Advice on Finless Porpoise Conservation]

  
\end{memo}




\newpage

%这一行是用来将Reference添加到目录的
\phantomsection\addcontentsline{toc}{section}{Refence}\tolerance=500

\bibliographystyle{apacite}
\bibliography{reference.bib}


\lhead{\small\sffamily \team}
\rhead{\small\sffamily Page \thepage\ }

\begin{appendices}

% \textbf{\textcolor[rgb]{0.98,0.00,0.00}{Input python source:}}
% \lstinputlisting[language=python]{./codes/verhulst.py}


% \textbf{\textcolor[rgb]{0.98,0.00,0.00}{Input matlab source:}}
% \lstinputlisting[language=Matlab]{./codes/lagrange_main.m}
% \lstinputlisting[language=Matlab]{./codes/lagrange.m}
% \lstinputlisting[language=Matlab]{./codes/ARIMA.m}
% \lstinputlisting[language=Matlab]{./codes/ca.m}
% \lstinputlisting[language=Matlab]{./codes/verhulst 2.m}

\end{appendices}


\end{document}

