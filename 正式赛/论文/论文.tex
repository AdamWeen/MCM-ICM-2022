\documentclass{mcmthesis}
\mcmsetup{CTeX = true,   % 使用 CTeX 套装时,设置为 true
        tcn = XJ162, problem = A,
        sheet = true, titleinsheet = true, keywordsinsheet = true,
        titlepage = false, abstract = false}
\geometry{left=0.75in,right=0.75in,top=1in,bottom=0.75in}
\numberwithin{figure}{section}
\numberwithin{table}{section}
\numberwithin{equation}{section}
\usepackage{newtxtext}
\usepackage{lipsum}
\usepackage{palatino}
\usepackage{hyperref}
\usepackage{booktabs}
\usepackage{subfigure}
\usepackage{graphicx}
\usepackage{pythonhighlight}
\usepackage{indentfirst}%首段自动缩进
\usepackage{colortbl}
\usepackage{apacite}
\usepackage{natbib}
\usepackage{tablefootnote}
\usepackage{multicol}
%算法四部曲↓
\usepackage{algorithm}
\usepackage{algpseudocode}
\usepackage{amsmath}
\usepackage{algorithmicx}
% \usepackage{threeparttable}

\setlength{\parindent}{2em}
\title{Finless Porpoise PVA based on Vortex, ARIMA, GF and CA}
\setlength{\headheight}{15pt}
\definecolor{darkOrange}{rgb}{0.929,0.49,0.192}
\definecolor{Orange}{rgb}{0.973,0.796,0.678}
\definecolor{lightOrange}{rgb}{0.988,0.894,0.839}
\definecolor{lightBlue}{rgb}{0.867,0.922, 0.969}
\definecolor{lightGreen}{rgb}{0.831,0.929,0.886}
\definecolor{lightYellow}{rgb}{1,0.949,0.8}
\begin{document}
\renewcommand{\algorithmicrequire}{\textbf{Input:}}  % Use Input in the format of Algorithm
\renewcommand{\algorithmicensure}{\textbf{Output:}} % Use Output in the format of Algorithm

\begin{abstract}



\begin{keywords}
\end{keywords}
\end{abstract}
\maketitle

\tableofcontents
  \thispagestyle{empty}
  \newpage
  \setcounter{page}{1}
%%
%%Generate the Memorandum, if it's needed.
%\memoto{\LaTeX{}studio}
%\memofrom{Liam Huang}
%\memosubject{Happy \TeX{}ing!}
%\memodate{\today}
%%\logo{\LARGE I'm pretending to be a LOGO!}
%\begin{memo}[Memorandum]
%  \lipsum[1-3]
%\end{memo}

\section{Introduction}

\subsection{Problem Restatement}



\subsection{Overview of Our Work}

\newpage



\section{Assumptions and Justifications}
These are necessary assumptions for simplifying the model.
\begin{enumerate}
  \item [1.] 
\end{enumerate}


\section{Notations}

\renewcommand\arraystretch{1.5}

\begin{table}[htpb!]
  \centering
  \caption{Notation Descriptions\citep{Vortex}} \label{Vortex_out}
  \begin{tabular}{m{2.5cm}<{\centering}|m{12.5cm}<{\centering}}
  \toprule[1.5pt]
  \textbf{Symbol} & \textbf{Definition} \\ \hline
  $ r $ & Innate rate of increase \\
  Stoch r & The mean population growth rate experienced in the simulations, averaged 
  across all years in which the population was extant. \\
  %内廪增长率:给定的物理和生物的条件下,具有稳定的年龄组配的种群的最大瞬时增长率
  N-extant  & Average extant population size\\ 
  N-all  & Average population size \\
  PE  & Probability of Extinction \\
  GeneDiv  &  Genetic Diversity \\
  TE & Time of Extinction(year) \\
  medianTE & If at least 50\% of the iterations went extinct, 
  the median time to extinction \\
  SD & Standard Deviation \\
  nAlleles & The mean number of alleles remaining within extant populations 
  (from an original number equal to twice the number of founder individuals)\\
  $ K $ & Carrying capacity \\ 
  $ N_t $ & Size of Finless Porpoise population in the year of $ 1991 + t $ \\ 
  \bottomrule[1.5pt]
  \end{tabular}
  \end{table}



\section{Introduction and Results of Models on Problem 1(a)}


\subsection{Result of Gray Forecast model}


\section{Sensitivity Test}

\section{Evaluation of Model}

\textbf{Strength:}


\section{Conclusions}



\newpage
\phantomsection\addcontentsline{toc}{section}{Policy Advice on Finless Porpoise Conservation}\tolerance=500
\memoto{Relavant Authorities}
\memofrom{MCM/ICM team XJ162}
\memodate{\today}

\begin{memo}[Policy Advice on Finless Porpoise Conservation]

  
\end{memo}




\newpage

%这一行是用来将Reference添加到目录的
\phantomsection\addcontentsline{toc}{section}{Refence}\tolerance=500

\bibliographystyle{apacite}
\bibliography{reference.bib}


\lhead{\small\sffamily \team}
\rhead{\small\sffamily Page \thepage\ }

\begin{appendices}

% \textbf{\textcolor[rgb]{0.98,0.00,0.00}{Input python source:}}
% \lstinputlisting[language=python]{./codes/verhulst.py}


% \textbf{\textcolor[rgb]{0.98,0.00,0.00}{Input matlab source:}}
% \lstinputlisting[language=Matlab]{./codes/lagrange_main.m}
% \lstinputlisting[language=Matlab]{./codes/lagrange.m}
% \lstinputlisting[language=Matlab]{./codes/ARIMA.m}
% \lstinputlisting[language=Matlab]{./codes/ca.m}
% \lstinputlisting[language=Matlab]{./codes/verhulst 2.m}

\end{appendices}


\end{document}

